\documentclass[online,helvetica]{chaksem}
\usepackage{color}
\usepackage[dvips]{graphicx}
\begin{document}

\begin{slide}
\heading{Better Living Through Statistics}
\heading{Monitoring Doesn't Have To Suck}

\center{\small
Jamie Wilkinson (jaq@spacepants.org)\\
Site Reliability Engineer, Google
\today
}
\end{slide}




I don't remember the first time I saw "#monitoringsucks" on Twitter, but it's
been a growing theme since then with several devops luminaries all chiming in
with their 70 characters/cent.

I can appreciate the sentiment.  I once used Nagios and tried to scale it.
However I can't agree that monitoring sucks.   Monitoring is awesome!
#monitoringsucks
It sure used to.

I remember using Nagios and cfengine to configure it.

Experimented with cacti.

Now I see collectd, graphite, and a world of new tools.
#monitoringsucks
What to talk about?

I have no idea what has gone on in the real world since then.

These #monitoringsucks people seem to be on the right path... do I have anything new to add?
Validation
http://blog.lusis.org/blog/2012/06/05/monitoring-sucking-just-a-little-bit-less/

"Instead of alerting on data and then storing it as an afterthought (perfdata anyone?) let’s start collecting the
data, storing it and then alerting based on it."


I am glad to have read this!
Why do we monitor?
To know if it's working

To know how well it's working

To be forewarned

To understand change

To plan for the future

We observe our systems to understand their behaviour.  Observation can be done
in many ad hoc ways, such as reading logs, or taking measurements.

We can use measurements of our systems to comparing against others (either other measures, or to the same measure over time.)

The process or result of determining or estimating ratios of quantities to
units.  To take a measurement, we need a base unit, a quantity, and some
repeatable measurement process to convert that quantity into a measure.

An important piece of the scientific method is to run an experiment to test a
hypothesis, and an experimental procedure must describe how one will measure
that change.

Let's also not forget that change introduces the unknown; the unknown by its
very nature can lead to nonworkingness.  Observation makes the unknown known,
and measurement turns the known into the understood.


Why do we monitor?
So the boss and the users aren't the first to tell you the site is down...
What is monitoring?
Measuring,
Recording,
Alerting,
Visualising

"If you can't measure it, you can't manage it."
As W. Edwards Deming says with regard to quality, "if you can't measure it, you can't manage it ..."If you can't measure it, you can't manage it!", is a famous quote of Robert Kaplan, the founder of the BSCas Floyd said, if you can’t measure it, you can’t manage it ... Remember Gordon Baskerville's famous forest management dictum: "If you can't measure it, you can't manage it ...a quote from George Webster that says "If you can't measure it, you can't manage it ...Grove is known for an almost ruthless analytic zeal (at Intel, one of his best-known maxims is "If you can't measure it, you can't manage it"),The cardinal rule of telecommunications: If you can't measure it, you can't manage it." T. TRAVERS WALTRIP ... as Peter Drucker is noted for saying, ‘if you can’t measure it, you can’t manage it ...... If you can't measure it, you can't manage it," Sugrue says... The famous quote attributed to Frederick W. Smith, the founder of Federal Express-"if you can't measure it, you can't manage it"AND PROBABLY THE MOST RELIABLE CITATION OF ALL...... If you can’t measure it, you can’t manage it.” The author of this quote is unknown, but it is quoted all the time.
Just some tangential lols.
Are you measuring?
Are you really, actually measuring?

What are you measuring?

How are you measuring?


Before you can fix your monitoring, you need to know if the data you're collecting is the right data.

Are you asking the right questions?
Monitoring systems
automate the boring parts:
Measuring,
Recording,
Alerting,Visualisation

so you have more time to do fun things... and debug the occasional emergency.

Already in our systems, we've automated the taking and recording of
measurements.  For example we have rrdtool and the tools built on top of that,
collectd, we have graphite, we have nagios' performance data extension to the
plugin protocol, and we have third party applications in the cloud offering to record
measurements.

The current state of monitoring:
The alerter
Measure, check against a threshold, fire an alert if check fails.

Measure
Record
Alert
Visualise



For example:
cucumber
nagios
icinga
OpenNMS

The current state of monitoring:
The rear view mirror
Measure, record into a timeseries database, show pretty charts.

Measure
Record
Alert
Visualise
For example:
MRTG
Cacti
ntop 
collectd
Ganglia
Munin

Problems with the check/alert model
Thresholds vary among instances, tuning difficult.
Adding new targets, new checks is lots of effort.
Checking logic performs the measurement and the "judgement" all in one.
Alerts for things you can't act on.
Application health
Self service

Need something more generic - more scalable?



Hard to configure and doesn't scale well.
An idea...
http://blog.lusis.org/blog/2012/06/05/monitoring-sucking-just-a-little-bit-less/

"Instead of alerting on data and then storing it as an afterthought (perfdata anyone?) let’s start collecting the
data, storing it and then alerting based on it."


How do we monitor?
To know if it's working

To know how well it's working

To be forewarned

To understand change

To plan for the future
How do we monitor?
To know if it's working - BLACKBOX

To know how well it's working - WHITEBOX

To be forewarned - WHITEBOX

To understand change - WHITEBOX

To plan for the future - WHITEBOX
Blackbox vs Whitebox
Blackbox: treat the system as opaque: you can only use it as a user would.

a.k.a. "probing"

c.f. cucumber-nagios

Flicking a lightbulb on to test you've installed it properly;                                                                                                  
                                                                                                                                                               
The "play a test sound" in the sound control panels on a computer or on your home theatre                                                                      
system;                                                                                                                                                        
                                                                                                                                                               
"print a test page" on a printer;                                                                                                                              
                                                                                                                                                               
trying to open the door after you've                                                                                                                           
just locked it to make sure it actually locked;                                                                                                                
                                                                                                                                                               
maybe pressing the "lock car"                                                                                                                                  
button on your car key again, just in case, because you can't remember if you                                                                                  
pressed it or if it worked but this time you look for the light flash...                                                                                       
                                                                                                                                                               
pretty much any everyday use of an object where you are not doing it for the actual                                                                            
outcome, just to test its working OK.                                                                                                                          
                                                                                                                                                               
"It looks a lot like just using the object."   

Blackbox vs Whitebox
Whitebox: expose the internal state of the system for inspection

a.k.a. instrumentation, telemetry,
... ROCKET SCIENCE

c.f. ... graphite? new relic? metrics?
Stovetop: clock                                                                                                                             
                                                                                                                                                               
Car dashboard: speedometer, tachometer, odometer, tripmeter, clock, fuel gauge
What's wrong with blackbox?
Only boolean: no visibility into why

Why is the site slow?
Why has image serving stopped working?

No predictive capability

How long until we need more disks? cpus? datacenters?

You know you need to get more fuel, but how long can you drive to get there?

I use the tripmeter to give me an estimate...
Pseudo-whitebox
Expose some internal state
Test the latest point in time against a threshold.
Fire an alert.



... ~same as probing
I assert that this is gives you the same information as the blackbox probe -- you don't get the internal state as a result, only the truthiness of the test.
Alerting on thresholds
Some real world examples; this one is a good match for the simple check/alert model.

https://www.youtube.com/watch?v=kn_dYZn5TEQ&feature=player_embedded

Alert when beer supply low
if cases - 1 - 1 <= 1
   Barney Worried About Beer Supply
time
cases
3
Barney gets worried
Disk full alert
Alert when 90% full
Different filesystems have different sizes
10% of 2TB is 200GB
False positive! 

Alert on absolute space, < 500MB
Arbitrary number
Different workloads with different needs 500MB might not be enough warning
Some alerts don't map well to the check/alert model.
Disk full alert
More generic alert:
How long before the disk is full?

How long will it take to respond to an (almost) full disk?
Alerting on rates of change 






src=https://www.youtube.com/embed/pfwmMfyPCB8?rel=0&start=71&end=85&autoplay=1

(skip to 1:11)

More complex real world example.


Dennis Hopper's alert
if speed > 50mph any time in the past
	Bomb Armed
if Bomb Armed and speed < 50mph ...
time
speed
50
armed
explodes
Keanu's alert
if speed > 50mph
Save the bus
time
speed
Save the bus!
First he wants to know what bus it is.
Keanu's alert
time
acceleration
speed
armed
explodes
0
inflection point
danger zone
it doesn't matter how long it takes to save thew bus as long as the bus stays above 50mph.  At what point does the rescue operation become time critical?

When the bus starts slowing down, acceleration goes negative.
Keanu's alert
speed - acceleration * time = 50
50 - speed = - acceleration * time
(speed - 50)/acceleration = time

if (speed - 50)/acceleration <= time to save bus
	Start saving the bus!
You don't necessarily care that acceleration is negative though -- but when it is, you do want to know if it's rapidly decelerating, because that changes how much time you have to finish the rescue.
New tools at our disposal
Calculus!

the derivative of speed -> acceleration

the derivative of acceleration -> ... jerk

(impulse?)


So timeseries are just curves.  We can apply high school calculus to them!

There's a few caveats, though; look close enough and your timeseries are discreet.
Error spike
error count
In this example, our errors per second rises, but we have no threshold.  Perhaps your threshold is 0?  What about user generated errors, like 404s?
Rate of errors vs normal rate
rate of change increases greater than expected
errors per second
Now we're looking at the rate of errors, we can set the threshold for alerting based on the noise floor of errors.
calculate rate of change of timeseries
Summary: New tool #1: calculate the rates of things and compare the rate against a threshold.
Another new tool
Not just looking at the latest data point, or the derivative at the latest point

Look back 5 minutes, 1 hour, 7 days, back to the dawn of time



The next new tool alluded to in the Speed example is historical data.  We have a whole time*series* available, so we don't have to limit our checks to only the most recent data point.  I know some checks store the last few data points for calculating rates or trends over the short term.  How many data points is enough?  With the timeseries database, you don't have to answer that question at the check level.
Traffic spike with threshold
worth getting out of bed for?
Let's say this is a rate of errors which we just started calculating.  Do you want to be paged for a problem that seems to have subsided?

Sure you might want to know about this issue for debugging, but was it necessary to wake you at 3am to tell you about it?
Δt
Traffic spike with threshold
worth getting out of bed for?
NO
maybe?
Δt
When an alerting condition arrives, why not wait for a bit and see if the condition is stable.  You do add a bit of latency to the alert (delta t at least before the alert fires) but you reduce your false positive rate and keep the oncall operator a little bit more sane, and their spouse happy!

Consider also flapping alerts.
observe timeseries history to gain context
Summary: New tool #2: Historical analysis.
Timeseries Have Types
Counter: monotonically nondecreasing
  "preserves the order" i.e. UP
  "nondecreasing" can be flat

So there's a few properties of timeseries that makes classifying them useful.

A counter indicates discreet events like the number of units of a measurand an action has been taken on
 the number of queries received, number of bytes transferred.


A counter can only indicate a base unit, such as distance, or time, or count of
queries, since a reference point.

Timeseries Have Types
Gauge: everything else... not monotonic


a gauge indicates a point in time, like a quantity of fuel remaining, a
length of a queue, or velocity.

A gauge can indicate derived units, such as revolutions per second, as well as
base units such as litres.


The difference is a counter is always increasing, whereas a gauge indicates a
measurand increase or decrease over time.



Counters FTW
Δt
Alas we cannot store every point in a timeseries; even though we said they're discreet so we don't need to store infinity points, we probably are still memory constrained, or CPU constrained, which affects the physical ability to collect the timeseries.  Depending on your requirements, perhaps you only need a point every minute or so.

The counter preserves information despite the loss of data by downsampling.  We know just how much a counter has increased since the last time we measured it, because we can guarantee that the counter hasn't increased any more than the value we've seen.
Counters FTW
no loss of meaning after sampling
Δt
Gauges FTL
Δt
The gauge, on the other hand, has no regard for the sampling interval, and will happily spike while you're not looking.

Gauges FTL
lose spike events shorter than sampling interval
Δt
So it's better in general to export your data as a counter, and perform your calculations on that base unit.
prefer counters over gauges
Gauges lose data when sampling.
Counters do not.
Prefer counters over gauges as base unit single dimension, easier to work with (and no missed spikes!)

Another new tool
Instances in a cluster don't work alone.

Discover the properties of the system by aggregating the parts.

How many queries per second is your cluster receiving?

Sum the query counters across the cluster, and calculate the rate!
Aggregation
cluster rate = rate(instance 1 + instance 2)
Timeseries are lists of points.  Assuming the sampling rate is the same along all of them, you can sum elements at the same position in each list together.  
aggregate to each logical grouping in the system
high school maths recap
Timeseries Operations: Rates
δ(counter)/δt = gauge

δ(gauge)/δt = gauge

(beware of sampling errors)

You change types of timeseries when applying some operations, for example taking the rate of a counter turns it into a gauge.  This is not bad as long as you remember to keep the counter around, as we'll see.

As an approximation, you can precalculate the rate at the end of each sampling interval by taking the delta between this and the last data point.

But beware of sampling errors; if you miss a collection, then there's no data to compare to.  Do you want to take the rate to the last valid data point if it was 10 minutes ago?
Timeseries Operations: Aggregation
Σ0..n(counter) = counter
Σ0..n(gauge) = gauge


(beware of sampling error and quantization)

To save time, precompute the sum just after each sample interval.

Remember the timeseries is actually discreet points in time; two timeseries probably don't align up on timestamps exactly.

Timeseries Operations: Ratios
counter / counter = counter: instant means
gauge / gauge = gauge: rate comparisons

e.g. New deployment
δ(errors) / δ(queries) > threshold?

Is the rate of rate of errors over rate of queries too high? :-)

Apply historical analysis again; instead of hardcoding a threshold, how about the mean error/query ratio rate for the lifetime of the last deployment?

Timeseries Operations: Mean
gauge = gauge*
counter = counter


What sort of mean?
Mean over N datapoints in single timeseries
Mean over N instances of same timeseries

Treat the group timeseries as a list of vectors; are you taking the mean along a single vector or along the same column?
Mean aggregations
Zone 1 has 5 tasks, zone 2 has 10
Both doing 100 qps

Mean 10 minute query rate?
Zone 1: 100/5 = 20
Zone 2: 100/10 = 10

Is the global average 15 qps per job?
Incorrect Mean
level 1 sum = sum(counters)
level 1 count = len(counters)
level 1 mean = level 1 sum / level 1 count

level 2 mean = sum(level 1 mean)/ len(level 1 mean)

(20 + 10)  / 2 = 15
The wrong way.
Correct Mean
level 1 sum = sum(counters)
level 1 count = len(counters)
level 1 mean = level 1 sum / level 1 count

level 2 sum = sum(level 1 sum)
level 2 count = sum(level 1 count)
level 2 mean = level 2 sum / level 2 count

(100 + 100) / (10 + 5) = 13.3

keep sums and counts at each level to prevent error
Sum counters before calculating rates at an aggregation level
Mean only defined over same denominator, keep count and sum at each aggregation level
What to measure?
All well and good, but you want me to give you concrete examples.
What to measure?
Not definitive, but a good place to start:

Queries per second
What is a query?
Errors per second
Type of error
Latency
by query type, response code, payload size
Bandwidth
by direction, query type, response code, ...
What not to measure
Load average..
What to alert on?
Rate of change of QPS outside normal cycles
Ratio of errors to queries
Latency (mean, 95th percentile) too high
Rate of change of bandwidth

Make sure it's ACTIONABLE...

then DOCUMENT IT
Caveats with timeseries based alerting
Timeseries alerting is a huge space

unlimited number of ways to break system
amount of logic necessary for high coverage is staggering
false positive rate must be low
alert tuning is time consuming
alert logic must be simple
Blackbox testing still necessary
End-to-end testing by definition covers everything you have missed,
you still have charts in the timeseries to inspect, right?
TLDL
Do maths on your timeseries (sums, rates)
Keep counters instead of gauges, derive rates
Compare them to one another (ratios)
Do historical analysis (compare values over time)
Alert only when action can be taken

Too long didn't listen

What next?
Attach a statistical package to your timeseries database, and experiment

R
numpy
Processing
your favourite here

Make smarter alerts!
I've talked about some of the mathematical techniques for analysing timeseries to be smarter about alerting

I can't give you the software we use internally, but

Here's some statistical tools you can use to start analysing your own data

If you can hook them up to your alerting system, then fantastic!  I encourage you all to experiment and hopefully come up with the next generation of monitoring tools!

Don't just stick to the operations I've suggested -- there's many statistical methods at your disposal I haven't mentioned, including percentiles and distribution functions, scatterplots and line fitting.

Any Questions!?

\end{document}
